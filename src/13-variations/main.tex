\documentclass[14pt]{extarticle} %Класс позволяет использовать базовые шрифты бОльших размеров
\usepackage[utf8x]{inputenc} %кодировка файла макета utf8
\usepackage[russian]{babel} 
\usepackage[left=25mm,right=15mm,top=20mm,bottom=20mm]{geometry} %Попытка разобраться с полями страниц
\usepackage{ntheorem} %окружение для настройки теорем 
\usepackage{graphicx} %работа с рисунками
\usepackage[labelsep=period,figurewithin=none,tablewithin=none]{caption} %подписи к объектам (рисунки, таблицы)
\usepackage{listings} %работа с листингами
\usepackage{indentfirst} %отступ первого абзаца в разделе
\usepackage{enumitem} %настройка маркированных и нумерованных списков (см. примеры настройки в тексте)
\usepackage{url} %формирование ссылок на электронные источники
\usepackage{fancyhdr} %Настройка нумерации страниц
\usepackage{tocloft} %Настройка заголовка для содержания

%====================================================================
% Мои настройки
\usepackage{amsmath}
\usepackage{amssymb}
\usepackage{booktabs}
\usepackage{caption}
\usepackage{subfig}
\usepackage{algorithm}
\usepackage{algorithmic}
\usepackage{bm}
\usepackage[para,online,flushleft]{threeparttable}
%\graphicspath{ {imgs/} }
\usepackage{xifthen}

%====================================================================

%====================================================================
%Настройки макета
%----------------
%Содержимое этого блока не должно подвергаться изменению
%====================================================================
\selectlanguage{russian}
\setlength{\parindent}{1.27cm}

%---------Размеры страницы
%\textwidth=11.3cm \textheight=17cm 
%\hoffset=-30mm \voffset=-15mm

%---------Настройка подписей к таблицам
\DeclareCaptionFormat{mplain}{#1#2\par \centering #3\par}
\captionsetup[table]{format=mplain,
justification=raggedleft,%
labelsep=none,%
singlelinecheck=false,%
skip=3pt}

%---------Настройка подписей к таблицам
%\setbeamertemplate{caption}[numbered]
%\setbeamertemplate{footline}[frame number]
\newcommand{\rulesep}{\unskip\ \vrule\ }
\newcommand{\source}[1]{\caption*{\textbf{Источник}: {#1}} }

%Настройка нумерации страниц
\fancyhf{} % clear all header and footers
\renewcommand{\headrulewidth}{0pt} % remove the header rule
\rfoot{\small \thepage}
\pagestyle{fancy}

%Настройка заголовка для содержания
\renewcommand{\cfttoctitlefont}{\hfill\normalfont\large\bfseries}
\renewcommand{\cftaftertoctitle}{\hfill\thispagestyle{empty}} 

%Настройка теорем
\theoremseparator{.}

%---------Команды рубрикации--------------

%Заголовки
\makeatletter
\renewcommand{\section}{\@startsection{section}{1}%
{\parindent}{-3.5ex plus -1ex minus -.2ex}%
{2.3ex plus.2ex}{\normalfont\large\bfseries}}

\renewcommand{\subsection}{\@startsection{subsection}{2}%
{\parindent}{-3.5ex plus -1ex minus -.2ex}%
{1.5ex plus.2ex}{\normalfont\large\bfseries}}

\renewcommand{\subsubsection}{\@startsection{subsubsection}{3}%
{\parindent}{-1.5ex plus -1ex minus -.2ex}%
{0.5ex plus.2ex}{\normalfont\bfseries}}
\makeatother

%Команда уровня главы
\newcommand{\mysection}[1]{
 \newpage
 \refstepcounter{section}
 {
  \section*{Глава \thesection. #1 \raggedright }
 }
 \addcontentsline{toc}{section}{Глава \thesection. #1} 
}

%Команда уровня параграфа
\newcommand{\mysubsection}[1]{
 \refstepcounter{subsection}
 \subsection*{\thesubsection. #1}
 \addcontentsline{toc}{subsection}{\thesubsection. #1}
}

%Команда третьего уровня
\newcommand{\mysubsubsection}[1]{
\refstepcounter{subsubsection}
% \addcontentsline{toc}{subsubsection}{\thesubsubsection. #1}
\subsubsection*{#1}
}

%Оформление Приложений
\newcounter{appendix}
\newcommand{\addappendix}[1]{
 \newpage
 \refstepcounter{appendix} 
 \section*{ПРИЛОЖЕНИЕ \theappendix. \\#1}
 \addcontentsline{toc}{section}{ПРИЛОЖЕНИЕ \theappendix. #1}
}

%Команда ненумерованной главы
\newcommand{\mynonumbersection}[1]{
\newpage
{
%	\begin{center}\section*{#1}\end{center}
	\centering\section*{#1}
}
\addcontentsline{toc}{section}{#1} 
}

%--------Настройка маркированных и нумерованных списков
\setlist{itemsep=0pt,topsep=0pt}

%--------Настройка листингов программного кода
\lstloadlanguages{C,[ANSI]C++}%!настройка листинга
%Можно подключить другие языки (см документацию к пакету)

%--------Тонкая настройка листингов
\lstset{
inputencoding=utf8x,
extendedchars=false,
showstringspaces=false,
showspaces=false,
keepspaces = true,
basicstyle=\small\ttfamily,
keywordstyle=\bfseries,
tabsize=2,                      % sets default tabsize to 2 spaces
captionpos=t,                   % sets the caption-position to bottom
breaklines=true,                % sets automatic line breaking
breakatwhitespace=true,        % sets if automatic breaks should only happen at whitespace
title=\lstname,                 % show the filename of files included with \lstinputlisting;
basewidth={0.5em,0.45em},
}

%----------Настройка подписей к листингам
\renewcommand{\lstlistingname}{Листинг}

%------------Подключение стиля для оформления списка литературы
\makeatletter
\renewcommand{\@biblabel}[1]{#1.\hfill}
\makeatother
\bibliographystyle{ugost2003s}
\PrerenderUnicode{ЙЦУКЕНГШЩЗХЪЭЖДЛОРПАВЫФЯЧСМИТЬБЮйцукенгшщзхъэждлорпавыфячсмитьбю}


%----------Use frames for text
\usepackage{mdframed}
\newcommand{\myframe}[1]{
\begin{mdframed}
\begin{center}
#1
\end{center}
\end{mdframed}
}

%----------Setup images
\captionsetup{figurename=Рисунок}

\newcommand{\myimage}[4]{ % source?; width; caption; img
\begin{figure}[h]
\centering
\includegraphics[width=#2\textwidth]{#4}
\caption{#3}
\ifthenelse{\equal{#1}{}}{
}{
\source{#1}
}
\end{figure}
}

%----------Setup Math stuff
\newtheorem{definition}{Определение}[section]
\newtheorem{proof}{Доказательство}
\newtheorem{theorem}{Теорема}

\makeatletter
\renewcommand*{\ALG@name}{Алгоритм}
\makeatother

\newcommand{\mydefinition}[1]{
\myframe{
\theoremstyle{definition}
\begin{definition}
#1
\end{definition}
}}

\newcommand{\myproof}[1]{
\theoremstyle{proof}
\begin{proof}
#1
\end{proof}
}

\newcommand{\mytheorem}[1]{
\myframe{
\theoremstyle{theorem}
\begin{theorem}
#1
\end{theorem}
}}

\newcommand{\myequation}[2][]{
\begin{equation}
\ifthenelse{\equal{#1}{}}{
\begin{aligned}#2\end{aligned}
}{
\left.\begin{aligned}#2\end{aligned}\right\} #1
}
\end{equation}
}

\graphicspath{ {imgs/} }

\begin{document}

%==================================================
\mynonumbersection{ВАРИАЦИОННОЕ ИСЧИСЛЕНИЕ}

%==================================================
\mysection{Введение}

%========================================
\mysubsection{Общая постановка}

В общем случае в задачах вариационного исчисления задается класс $D$ кривых или гиперповерхностей, описываемых функциями $x : G \subseteq R^n → R^m.$ Как правило, предполагается достаточная гладкость $x(s), s \in G$, а также накладываются некоторые дополнительные требования, в частности, на границе $G$. Далее на элементах $x(s)$ класса $D$ определяется интегральный функционал 

\myequation{
J[x] = \int_G F(s, x(s), . . .)ds
}
\

подынтегральная функция которого может зависеть не только от $s$ и $x(s)$, но и от производных $x(s)$ до некоторого порядка. Задача заключается в определении $x^* \in D$ для которого

\myequation{
J[x^*] = \min_{x \in D} J[x]
}
\

если глобальный минимум существует, или в определении локального минимума, тип которого зависит от вводимого понятия близости элементов в $D$. Кроме задач на минимум могут рассматриваться задачи на максимум.

%========================================
\mysubsection{Примеры задач вариационного исчисления}

%==============================
\mysubsubsection{Задача с фиксированными концами}

Пусть $G = [t_0, t_1]$, а множество допустимых кривых $D$ имеет вид:

\myequation{
D = D_1 = \{ x : x \in C^2(G), x : G \rightarrow R^1, x(t_0) = x_0, x(t_1) = x_1 \}
}
\

\myequation{
J[x] = \int^{t_1}_{t_0} F(t, x(t), \dot{x}(t))dt
}
\

где функция $F$ предполагается непрерывной вместе со своими частными производными до второго порядка по совокупности всех переменных.

%==============================
\mysubsubsection{Задача со свободными концами}

Пусть $G = [t_0, t_1]$, а множество допустимых кривых $D$ имеет вид:

\myequation{
D = D_2 = \{ x : x \in C^2(G), x : G \rightarrow R^1 \}
}
\

функционал $J[x]$ имеет вид (4) с теми же требованиями к функции $F$, что и в предыдущей задаче. Как видно из (5) ограничений на значения кривой на концах промежутка интегрирования, в отличие от задачи с фиксированными концами, не накладывается.

%==============================
\mysubsubsection{Задача со скользящими концами}

Пусть $G = (-\infty, \infty)$ и заданы граничные кривые, определяемые соотношениями

\myequation{
\Psi_i(x, t) = 0, (i = 0, 1)
}
\

функции $\Psi_i$ считаются достаточно гладкими по совокупности переменных.

Множество допустимых кривых $D$ имеет вид:


\myequation{
D = D_3 = \{ x : x \in C^2(G), x : G \rightarrow R^1; \exists t^x_i : \Psi_i(x(t^x_i ), t^x_i ) = 0, \\
(i = 0, 1); \dot{x}(t^x_i) \neq − (\Psi'_{it}(x(t^x_i ), t^x_i ) / \Psi'_{ix}(x(t^x_i ), t^x_i )) \}
}
\


\myequation{
J[x] = \int^{t_1^x}_{t_0^x} F(t, x(t), \dot{x}(t))dt
}
\

где на функцию $F$ накладываются прежние требования и $t^x_0 < t^x_1$ -- значения $t$, соответствующие пересечениям без касания кривой $x = x(t)$ с граничными кривыми.

Заметим, что в (8) пределы интегрирования зависят от кривой $x = x(t)$ и определяются ближайшими друг к другу точками её пересечения с граничными кривыми (6). Последнее условие в виде запрета равенства для производной $\dot{x}(t^x_i)$ в определении $D_3$ означает отсутствие касания кривой $x = x(t)$ с граничными кривыми в точках пересечения. Это требование необходимо для того, чтобы точка пересечения не исчезала при достаточно малом изменении кривой $x = x(t)$.

%==================================================
\mysection{Слабый и сильный локальный минимумы}

\mydefinition{
Локальным минимумом функционала $J$ из (4) на множестве кривых $D$ из (3) или (5) пространства $B$ назовем такую кривую $x^o \in D$ для которой $\exists \epsilon > 0$, что $\forall x \in D \cap O_\epsilon(x^o)$ выполняется

\myequation{
J[x^o] \leq J[x]
}
\

где

\myequation{
O_\epsilon(x^o) = \{ x \in B : || x^o − x || \leq \epsilon \} .
}
\
}

В зависимости от способа задания нормы можно получить различные трактовки понятия «локальный минимум». Введем нормы двух типов.

\mydefinition{
Норму вида
\myequation{
|| x ||_0 = \max_{t \in [t_0, t_1]} | x(t) |
}
\
определенную на линейном пространстве непрерывных на $[t0, t1]$ функций $(x ∈ C[t0, t1])$ назовем нормой нулевого порядка.
}
\

\mydefinition{
Норму вида
\myequation{
|| x ||_1 = \max_{t \in [t_0, t_1]} | x(t) | + \max_{t \in [t_0, t_1]} | \dot{x}(t) |
}
\
определенную на линейном пространстве непрерывно дифференцируемых на $[t0, t1]$ функций $(x ∈ C^1[t0, t1])$ назовем нормой первого порядка.
}
\

Аналогично можно ввести понятие нормы $k$-го порядка.

\mydefinition{
Локальный минимум, понимаемый в смысле нормы нулевого порядка из (11) называется сильным локальным минимумом функционала, а понимаемый в смысле нормы первого порядка из (12) — слабым локальным минимумом.
}
\

Заметим, что всякий сильный локальный минимум одновременно является и слабым, однако обратное неверно.

%==================================================
\mysection{Метод вариаций Лагранжа}

Пусть выделена некоторая допустимая кривая $\hat{x} \in D$. Определим для нее класс пробных функций $M(\hat{x})$:

\myequation{
M(\hat{x}) = \{ \eta : \text{где } \eta : G \rightarrow R^1 \text{ и }  \forall \alpha \in R^1, \\
\text{если } |\alpha| \text{ -- достаточно мало, то } \hat{x} + \alpha \cdot \eta \in D \}
}
\

Таким образом, при $\eta \in M(\hat{x})$ и достаточно малом по модулю $\alpha$ изменение кривой $\hat{x}(t)$ с помощью добавки $\alpha \cdot \eta(t)$ не выводит измененную кривую $\hat{x} + \alpha \cdot \eta$ из допустимого множества $D$.

Изменение, добавляемое к функции $\hat{x}$, называют вариацией кривой $\delta \hat{x}$. В нашем случае $\delta \hat{x} = \alpha \cdot \eta$. Кривую $x = \hat{x} + \delta \hat{x}$ называют проварьированной кривой.

При использовании параметрических вариаций вида  $\delta \hat{x} = \alpha \cdot \eta$ значение функционала $J$ на проварьированной кривой может быть рассмотрено как функция, зависящая от параметра вариации $\alpha$:

\myequation{
Q(\alpha) = J[x + \alpha \eta]
}
\

Сделанные ранее предположения о задаче позволяют представить приращение функционала при варьировании кривой в виде разложения:

\myequation{
J[x + \alpha \eta] - J[x] = Q'(0) \alpha + Q''(0) \frac{\alpha^2}{2} + o(\alpha^2)
}
\

\mydefinition{
Первой вариацией функционала на кривой $x$ в направлении пробной функции $\eta \in M(x)$ назовем главную линейную часть приращения функционала:

\myequation{
\delta_\eta J(x, \alpha) = Q'(0) \cdot \alpha
}
\

Второй вариацией функционала назовем главную квадратичную часть его приращения:

\myequation{
\delta^2_{\eta \eta} J(x, \alpha) = Q''(0) \cdot \frac{\alpha^2}{2}
}
}
\

\mytheorem{
Если $x^o$ — локальный минимум функционала (слабый или сильный), то верно следующее:
\begin{enumerate}
  \item $\forall \eta \in M(x^o) : \delta_\eta J(x^o, \alpha) = 0$ -- необходимое условие первого порядка;
  \item $\forall \eta \in M(x^o) : \delta^2_{\eta \eta} J(x^o, \alpha) = 0$ -- необходимое условие второго порядка.
\end{enumerate}
}
\

\myproof{
Пусть это не так и первое утверждение неверно. Тогда найдется пробная функция $\hat{\eta} \in M(x^o)$, что $Q'(0) \neq 0$. Для определенности можно считать, что $Q'(0) < 0$. Тогда из (15) следует, что при любых достаточно
малых значениях $\alpha > 0$ выполнится

\myequation{
J[x^o + \alpha \hat{\eta}] < J[x^o]
}
\

Однако

\myequation{
|| x^o + \alpha \hat{\eta} - x^o||_1 = \alpha || \eta ||_1 \rightarrow 0
}
\

при $\alpha \rightarrow 0$, что означает принадлежность кривой $x^o + \alpha \hat{\eta}$ любой слабой окрестности $x^o$ при достаточно малом $\alpha$. Это противоречит слабой оптимальности кривой $x^o$. Таким образом, первое утверждение доказано для слабого локального минимума. Однако, поскольку сильный локальный минимум одновременно является слабым, то утверждение 1 справедливо и для сильного локального минимума.

Аналогично доказывается второе утверждение.
}
\

\mydefinition{
Экстремалями функционала $J$ (стационарными кривыми) называют такие допустимые кривые $\bar{x} \in D$ для которых

\myequation{
\forall \eta \in M(\bar{x}) : \delta_\eta J(\bar{x}, \alpha) = 0
}
}
\

\mytheorem{
\textbf{Лемма Лагранжа.} Пусть $f(t) \in C[t_0, t_1]$, т.е. $f$ -- непрерывна на отрезке, и

\myequation{
M = \{ \eta : \eta \in C^k[t_0, t_1], (k \geq 2), \eta(t_0) = \eta(t_1) = 0 \}
}
\

Если при этом $\forall \eta \in M$:

\myequation{
\int_{t_0}^{t_1} f(t) \eta(t) dt = 0
}
\

то $f(t) \equiv 0$ для $t \in [t_0, t_1]$.
}

\myproof{
Пусть утверждение неверно, т.е. $\exists \bar{t} \in [t_0, t_1]$, что $f(\bar{t}) = c \neq 0$ (не ограничивая общности будем считать, что $c > 0$). При этом (из непрерывности $f$) следует, что такое $\bar{t}$ найдется и в $(t_0, t_1)$. Тогда существует $\delta$–окрестность $\bar{t}$ в которой $f(t) > c/2$.

Построим функцию $\bar{\eta}(t)$, гладкую до порядка $k > 2$, чтобы

\myequation{
\bar{\eta}(t) = 
\begin{cases}
1, t \in O_{\delta / 2} (\bar{t}), \\
0, t \notin O_{\delta} (\bar{t}),
\end{cases}
}
\

и при этом была обеспечена неотрицательность $\bar{\eta}(t)$ для всех значений аргумента. Построенная функция $\bar{\eta}(t) \in M$.

Справедливы оценки

\myequation{
\int_{t_0}^{t_1} f(t) \bar{\eta}(t) dt = \int_{O_{\delta} (\bar{t})} f(t) \bar{\eta}(t) dt \geq c/2 \int_{O_{\delta / 2} (\bar{t})} \bar{\eta}(t) dt = \frac{c}{2} \delta > 0
}
\

Возникшее противоречие доказывает, что лемма верна.
}

%==================================================
\mysection{Необходимые условия оптимальности}

%========================================
\mysubsection{Условия оптмиальности первого порядка}

\mytheorem{
(условия оптимальности первого порядка для задачи с закрепленными концами). Для того, чтобы в задаче с закрепленными концами кривая $x = x^*(t) \in C^2 [t_0, t_1]$ являлась экстремалью функционала (4) необходимо и достаточно, а для того, чтобы она являлась экстремумом (минимумом или максимумом) необходимо, чтобы функция $x = x^*(t)$:
\begin{enumerate}
  \item являлась решением дифференциального уравнения Эйлера:
  \myequation{
  \frac{\delta F}{\delta x} (t, x, \dot{x}) - \frac{d}{dt} \big( \frac{\delta F}{\delta \dot{x}} (t, x, \dot{x}) \big) = 0
  }
  \item удовлетворяла граничным условиям:
  \myequation{
  x^*(t_i) = x_i, (i = 0, 1)
  }
\end{enumerate}
}
\

\mytheorem{
(условия оптимальности первого порядка для задачи со свободными концами). Для того, чтобы в задаче с закрепленными концами кривая $x = x^*(t) \in C^2 [t_0, t_1]$ являлась экстремалью функционала (4) необходимо и достаточно, а для того, чтобы она являлась экстремумом (минимумом или максимумом) только необходимо, чтобы функция $x = x^*(t)$:
\begin{enumerate}
  \item являлась решением дифференциального уравнения Эйлера (25)
  \item удовлетворяла так называемым естественным граничным условиям
  \myequation{
  \frac{\delta F}{\delta \dot{x}} (t_i, x(t_i), \dot{x}(t_i)) = 0, (i = 0, 1)
  }
\end{enumerate}
}
\

\mytheorem{
(условия оптимальности первого порядка для задачи со скользящими концами). Для того, чтобы в задаче со свободными концами кривая $x = x^*(t) \in C^2 [t_0, t_1]$, имеющая неособые пересечения с граничными кривыми, являлась экстремалью необходимо и достаточно, а для того, чтобы она являлась экстремумом (минимумом или максимумом) только необходимо, чтобы функция $x = x^*(t)$:
\begin{enumerate}
  \item являлась решением дифференциального уравнения Эйлера (25)
  \item выполнялись так называемые граничные условия трансверсальности
  \myequation{
  \frac{\delta F}{\delta \dot{x}} (t_i, x_i, \dot{x_i}) - \frac{ \Psi'_{ix}(x_i, t_i) F(t_i, x_i, \dot{x_i}) }{.\Psi'_{it}(x_i, t_i) + \dot{x}_i \Psi'_{ix}(x_i, t_i) } = 0, (i = 0, 1)
  }
\end{enumerate}
}

\end{document}

