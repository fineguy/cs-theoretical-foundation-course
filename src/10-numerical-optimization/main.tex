\documentclass[14pt]{extarticle} %Класс позволяет использовать базовые шрифты бОльших размеров
\usepackage[utf8x]{inputenc} %кодировка файла макета utf8
\usepackage[russian]{babel} 
\usepackage[left=25mm,right=15mm,top=20mm,bottom=20mm]{geometry} %Попытка разобраться с полями страниц
\usepackage{ntheorem} %окружение для настройки теорем 
\usepackage{graphicx} %работа с рисунками
\usepackage[labelsep=period,figurewithin=none,tablewithin=none]{caption} %подписи к объектам (рисунки, таблицы)
\usepackage{listings} %работа с листингами
\usepackage{indentfirst} %отступ первого абзаца в разделе
\usepackage{enumitem} %настройка маркированных и нумерованных списков (см. примеры настройки в тексте)
\usepackage{url} %формирование ссылок на электронные источники
\usepackage{fancyhdr} %Настройка нумерации страниц
\usepackage{tocloft} %Настройка заголовка для содержания

%====================================================================
% Мои настройки
\usepackage{amsmath}
\usepackage{amssymb}
\usepackage{booktabs}
\usepackage{caption}
\usepackage{subfig}
\usepackage{algorithm}
\usepackage{algorithmic}
\usepackage{bm}
\usepackage[para,online,flushleft]{threeparttable}
%\graphicspath{ {imgs/} }
\usepackage{xifthen}

%====================================================================

%====================================================================
%Настройки макета
%----------------
%Содержимое этого блока не должно подвергаться изменению
%====================================================================
\selectlanguage{russian}
\setlength{\parindent}{1.27cm}

%---------Размеры страницы
%\textwidth=11.3cm \textheight=17cm 
%\hoffset=-30mm \voffset=-15mm

%---------Настройка подписей к таблицам
\DeclareCaptionFormat{mplain}{#1#2\par \centering #3\par}
\captionsetup[table]{format=mplain,
justification=raggedleft,%
labelsep=none,%
singlelinecheck=false,%
skip=3pt}

%---------Настройка подписей к таблицам
%\setbeamertemplate{caption}[numbered]
%\setbeamertemplate{footline}[frame number]
\newcommand{\rulesep}{\unskip\ \vrule\ }
\newcommand{\source}[1]{\caption*{\textbf{Источник}: {#1}} }

%Настройка нумерации страниц
\fancyhf{} % clear all header and footers
\renewcommand{\headrulewidth}{0pt} % remove the header rule
\rfoot{\small \thepage}
\pagestyle{fancy}

%Настройка заголовка для содержания
\renewcommand{\cfttoctitlefont}{\hfill\normalfont\large\bfseries}
\renewcommand{\cftaftertoctitle}{\hfill\thispagestyle{empty}} 

%Настройка теорем
\theoremseparator{.}

%---------Команды рубрикации--------------

%Заголовки
\makeatletter
\renewcommand{\section}{\@startsection{section}{1}%
{\parindent}{-3.5ex plus -1ex minus -.2ex}%
{2.3ex plus.2ex}{\normalfont\large\bfseries}}

\renewcommand{\subsection}{\@startsection{subsection}{2}%
{\parindent}{-3.5ex plus -1ex minus -.2ex}%
{1.5ex plus.2ex}{\normalfont\large\bfseries}}

\renewcommand{\subsubsection}{\@startsection{subsubsection}{3}%
{\parindent}{-1.5ex plus -1ex minus -.2ex}%
{0.5ex plus.2ex}{\normalfont\bfseries}}
\makeatother

%Команда уровня главы
\newcommand{\mysection}[1]{
 \newpage
 \refstepcounter{section}
 {
  \section*{Глава \thesection. #1 \raggedright }
 }
 \addcontentsline{toc}{section}{Глава \thesection. #1} 
}

%Команда уровня параграфа
\newcommand{\mysubsection}[1]{
 \refstepcounter{subsection}
 \subsection*{\thesubsection. #1}
 \addcontentsline{toc}{subsection}{\thesubsection. #1}
}

%Команда третьего уровня
\newcommand{\mysubsubsection}[1]{
\refstepcounter{subsubsection}
% \addcontentsline{toc}{subsubsection}{\thesubsubsection. #1}
\subsubsection*{#1}
}

%Оформление Приложений
\newcounter{appendix}
\newcommand{\addappendix}[1]{
 \newpage
 \refstepcounter{appendix} 
 \section*{ПРИЛОЖЕНИЕ \theappendix. \\#1}
 \addcontentsline{toc}{section}{ПРИЛОЖЕНИЕ \theappendix. #1}
}

%Команда ненумерованной главы
\newcommand{\mynonumbersection}[1]{
\newpage
{
%	\begin{center}\section*{#1}\end{center}
	\centering\section*{#1}
}
\addcontentsline{toc}{section}{#1} 
}

%--------Настройка маркированных и нумерованных списков
\setlist{itemsep=0pt,topsep=0pt}

%--------Настройка листингов программного кода
\lstloadlanguages{C,[ANSI]C++}%!настройка листинга
%Можно подключить другие языки (см документацию к пакету)

%--------Тонкая настройка листингов
\lstset{
inputencoding=utf8x,
extendedchars=false,
showstringspaces=false,
showspaces=false,
keepspaces = true,
basicstyle=\small\ttfamily,
keywordstyle=\bfseries,
tabsize=2,                      % sets default tabsize to 2 spaces
captionpos=t,                   % sets the caption-position to bottom
breaklines=true,                % sets automatic line breaking
breakatwhitespace=true,        % sets if automatic breaks should only happen at whitespace
title=\lstname,                 % show the filename of files included with \lstinputlisting;
basewidth={0.5em,0.45em},
}

%----------Настройка подписей к листингам
\renewcommand{\lstlistingname}{Листинг}

%------------Подключение стиля для оформления списка литературы
\makeatletter
\renewcommand{\@biblabel}[1]{#1.\hfill}
\makeatother
\bibliographystyle{ugost2003s}
\PrerenderUnicode{ЙЦУКЕНГШЩЗХЪЭЖДЛОРПАВЫФЯЧСМИТЬБЮйцукенгшщзхъэждлорпавыфячсмитьбю}


%----------Use frames for text
\usepackage{mdframed}
\newcommand{\myframe}[1]{
\begin{mdframed}
\begin{center}
#1
\end{center}
\end{mdframed}
}

%----------Setup images
\captionsetup{figurename=Рисунок}

\newcommand{\myimage}[4]{ % source?; width; caption; img
\begin{figure}[h]
\centering
\includegraphics[width=#2\textwidth]{#4}
\caption{#3}
\ifthenelse{\equal{#1}{}}{
}{
\source{#1}
}
\end{figure}
}

%----------Setup Math stuff
\newtheorem{definition}{Определение}[section]
\newtheorem{proof}{Доказательство}
\newtheorem{theorem}{Теорема}

\makeatletter
\renewcommand*{\ALG@name}{Алгоритм}
\makeatother

\newcommand{\mydefinition}[1]{
\myframe{
\theoremstyle{definition}
\begin{definition}
#1
\end{definition}
}}

\newcommand{\myproof}[1]{
\theoremstyle{proof}
\begin{proof}
#1
\end{proof}
}

\newcommand{\mytheorem}[1]{
\myframe{
\theoremstyle{theorem}
\begin{theorem}
#1
\end{theorem}
}}

\newcommand{\myequation}[2][]{
\begin{equation}
\ifthenelse{\equal{#1}{}}{
\begin{aligned}#2\end{aligned}
}{
\left.\begin{aligned}#2\end{aligned}\right\} #1
}
\end{equation}
}

\graphicspath{ {imgs/} }

\begin{document}

%==================================================
\mynonumbersection{ЧИСЛЕННЫЕ МЕТОДЫ ОПТИМИЗАЦИИ}

%==================================================
\mysection{Численные методы оптимизации}

%========================================
\mysubsection{Основные определения}

Пусть задано множество $Х$ и функция $f(x)$ , определенная на этом множестве. Требуется найти точки минимума или максимума функции $f(x)$ на множестве $Х$. Задачу на минимум записывают следующим образом:

\myequation{
f(x) \rightarrow \min, x \in X
}
\

При этом функцию $f(x)$ называют целевой функцией; множество $Х$ – допустимым множеством; любое $x \in X$ -- допустимой точкой.

Ниже будем рассматривать так называемые конечномерные задачи оптимизации, то есть задачи допустимое множество $Х$ которых является подмножеством евклидового пространства $E^n$.

\mydefinition{
Точка $x^* \in X$ называется:
\begin{itemize}
  \item точкой глобального минимума функции $f(x)$ на множестве $Х$, если
  \myequation{
  f(x^*) \leq f(x) \; \forall x \in X
  }
  \item точкой локального минимума функции $f(x)$ на множестве $Х$, если
  \myequation{
  f(x^*) \leq f(x) \; \forall x \in X \cap U_\epsilon (x^*)
  }
\end{itemize}
}
\

Отметим, что глобальный минимум всегда одновременно является локальным, но не наоборот.

Если в (2) и (3) при $x \neq x^*$ имеем строгие неравенства, то точка $x^*$ называется соответственно точкой строгого глобального минимума, строгого локального минимума.

\mydefinition{
Вектор $h$ называется направлением убывания функции $f(x)$ в точке $x$, если выполняется

\myequation{
f(x + \alpha h) < f(x)
}
\

при всех достаточно малых $\alpha > 0$.
}
\

Множество всех направлений убывания функции $f(x)$ в точке $x$ обозначим $U(x, f )$

\mytheorem{
Пусть функция $f(x)$, дифференцируемая в точке $x \in E^n$. Если вектор $h$ является направлением убывания функции $f(x)$ в точке $x$, то справедливо

\myequation{
\langle \nabla f(x), h \rangle \leq 0
}
\

Если при некотором $h \in E^n$ выполняется

\myequation{
\langle \nabla f(x), h \rangle < 0
}
\

тогда вектор $h$ является направлением убывания функции $f(x)$ в точке $x$.
}
\

\myproof{
\textbf{Необходимость.} Пусть вектор $h \in U(x, f)$ и при этом $\langle \nabla f(x), h \rangle > 0$. В этом случае имеем

\myequation{
f(x + \alpha h) - f(x) = \langle \nabla f(x), \alpha h \rangle + O(\alpha) = \alpha \bigg( \langle \nabla f(x), h \rangle + \frac{O(\alpha)}{\alpha} \bigg) > 0
}
\

Здесь неравенство следует из того, что при достаточно малых $\alpha > 0$ знак выражения определяется знаком первого слагаемого. Таким образом, получили противоречие c (4), что и доказывает справедливость (5)

\textbf{Достаточность.} Пусть выполняется (6). В этом случае

\myequation{
f(x + \alpha h) - f(x) = \alpha \bigg( \langle \nabla f(x), h \rangle + \frac{O(\alpha)}{\alpha} \bigg) < 0
}
\

при всех достаточно малых $\alpha > 0$. Следовательно, вектор $h$ является направлением убывания функции $f(x)$ в точке $x$, то есть $h \in U(x, f)$.

}
\

\mydefinition{
Вектор $h \in E^n$ задает возможное направление относительно множества $Х$ в точке $x \in X$, если $x + \alpha h \in X$ при всех достаточно малых $\alpha > 0$ . Вектор $h$ в этом случае будем называть возможным направлением в точке x относительно множества $X$. Множество всех таких векторов $h$ обозначим через $V(x, X)$ (множество возможных направлений в точке $x \in X$).
}
\

\mytheorem{
(необходимое условие локальной оптимальности). Если $x^*$ -- локальное решение задачи (1), то

\myequation{
U(x^*, f) \cap V(x^*, X) = \O
}
}
\

\myproof{
Пусть $x^* \in X$ локальное решение (1) и при этом (9) неверно, то есть существует вектор $h \in E^n$ такой, что $f(x^* + \alpha h) < f(x^*)$ и при этом $x^* + \alpha h \in X$ при всех достаточно малых $\alpha > 0$. А это означает, что в любой сколь угодно малой окрестности точки $x^*$ существует точка $x = x^* + \alpha h \in X \cap U_\epsilon (x^*)$ такая, что справедливо $f(x) < f(x^*)$, что противоречит определению локального минимума.
}

%==================================================
\mysection{Методы безусловной оптимизации}

Рассмотрим задачу оптимизации:

\myequation{
f(x) \rightarrow \min, x \in E^n
}
\

\mytheorem{
(необходимое условие). Пусть функция $f(x)$ является дифференцируемой в точке $x^* \in E^n$. Если $x^*$ -- локальное решение задачи (10), то

\myequation{
\nabla f(x^*) = 0
}
}
\

\mydefinition{
Точка $x^*$, удовлетворяющая условию (11), называется стационарной точкой функции $f(x)$.
}
\

\mytheorem{
(необходимое условие оптимальности 2-го порядка). Пусть функция $f(x)$ дважды дифференцируемая в точке $x^* \in E^n$. Если $x^*$ -- локальное решение задачи (10), то матрица Гессе функции $f(x)$ в точке $x^*$ неотрицательно определена, то есть

\myequation{
\langle \nabla^2 f(x^*)h, h \rangle \geq 0 \; \forall h \in E^n
}
}
\

\mytheorem{
(достаточное условие). Пусть функция $f(x)$ дважды дифференцируемая в точке $x^* \in E^n$. И пусть в этой точке выполняется условие стационарности (11) и матрица Гессе положительно определена, то есть

\myequation{
\langle \nabla^2 f(x^*)h, h \rangle > 0 \; \forall h \in E^n
}
\

Тогда $x^*$ -- строгое локальное решение задачи (10).
}
\

%========================================
\mysubsection{Метод наискорейшего спуска}

Общая схема методов спуска, в которых последовательность приближений $x^1, x^2, \ldots$ к точке минимума строится по правилу:

\myequation{
x^{k+1} = x^k + \alpha_k h^k
}
\

где направление $h^k$ принадлежит множеству направлений убывания функции $h^k \in U(x^k, f); \alpha_k \geq 0$ -- параметр, определяющий длину шага вдоль направления $h^k$.

В градиентных методах направление $h^k$ берется равным антиградиенту функции $f(x)$ в точке $x^k$, то есть $h^k = - \nabla f(x^k)$. В градиентных методах используются различные методы выбора шага $\alpha_k$. Если длина шага выбирается из минимизации функции вдоль направления антиградиента, то получаем вариант градиентного метода, называемый методом наискорейшего спуска.

Итак, в методе наискорейшего спуска шаг $\alpha_k$ вдоль направления $h^k$ выбирается из решения оптимизационной задачи:

\myequation{
f(x^k - \alpha_k \nabla f(x^k)) = \min_{\alpha \geq 0} f(x^k - \alpha_k \nabla f(x^k))
}
\

\mytheorem{
Пусть функция $f(x)$ дифференцируема на всем пространстве $E^n$, и ее градиент удовлетворяет условию Липшица:

\myequation{
|| \nabla f(x + \Delta x) - \nabla f(x) || \leq L || \Delta x ||, \; \forall x, x + \Delta x \in E^n
}
\

Тогда для для остаточного члена в разложении $\Delta f(x) = \langle \nabla f(x), \Delta x \rangle + O(|| \Delta x ||)$ справедлива оценка:

\myequation{
O(|| \Delta x ||) \leq \frac{L}{2} || \Delta x || ^ 2
}
}
\

\myproof{
Для любых $x, x + \Delta x \in E^n$ справедливо соотношение

\myequation{
\Delta f(x) = f(x + \Delta x) - f(x) = \int_0^1 \langle \nabla f(x + \alpha \Delta x), \Delta x \rangle d \alpha
}
\

где $\alpha \in [0, 1]$. Действительно, рассмотрим функцию переменной $\alpha : g(\alpha) = f(x + \alpha \Delta x)$. Её производная по переменной $\alpha$ имеет следующий вид:

\myequation{
\frac{d g}{d \alpha} (\alpha) = \langle \nabla f(x + \alpha \Delta x), \Delta x \rangle
}
\

а отсюда следует справедливость

\myequation{
\int_0^1 \frac{d g}{d \alpha} (\alpha) = g(1) - g(0) = f(x + \Delta x) - f(x)
}
\

Таким образом можем представить

\myequation{
\Delta f(x) = \int_0^1 \langle \nabla f(x + \alpha \Delta x), \Delta x \rangle d \alpha = \\
=  \int_0^1 \langle \nabla f(x), \Delta x \rangle d \alpha + \int_0^1 \langle \nabla f(x + \alpha \Delta x) - \nabla f(x), \Delta x \rangle d \alpha \leq \\
\leq \int_0^1 \langle \nabla f(x), \Delta x \rangle d \alpha + \int_0^1  || f(x + \alpha \Delta x) - \nabla f(x) || \cdot || \Delta x || d \alpha
}
\

Так как градиент функции $f(x)$ удовлетворяет условию Липшица

\myequation{
\Delta f(x) \leq \langle \nabla f(x), \Delta x \rangle + \int_0^1  || f(x + \alpha \Delta x) - \nabla f(x) || \cdot || \Delta x || d \alpha \leq \\
\leq \langle \nabla f(x), \Delta x \rangle + L \int_0^1  || \alpha \Delta x || \cdot || \Delta x || d \alpha = \langle \nabla f(x), \Delta x \rangle + L || \Delta x ||^2 \int_0^1  \alpha d \alpha = \\
= \langle \nabla f(x), \Delta x \rangle + \frac{L}{2} || \Delta x ||^2 
}
\

Таким образом имеем

\myequation{
\Delta f(x) = \langle \nabla f(x), \Delta x \rangle + O(|| \Delta x ||) \leq \langle \nabla f(x), \Delta x \rangle + \frac{L}{2} || \Delta x ||^2
}
}

%========================================
\mysubsection{Метод Ньютона}

Метод Ньютона решения задач безусловной минимизации относится к методам 2-го порядка, то есть к методам, используемым информацию о вторых производных целевой функции $f(x)$.

Предположим, что функция $f(x)$ дважды непрерывно дифференцируема в $E^n$. Пусть начальное приближение $x^0 \in E^n$ задано и с помощью метода Ньютона уже найдено $k$-ое приближение $x^k \in E^n$ . В некоторой окрестности точки $x^k$ функцию $f(x)$ аппроксимируем квадратичной функцией $\psi_k(x)$:

\myequation{
\psi_k(x) = f(x^k) + \langle \nabla f(x^k), x - x^k \rangle + \frac{1}{2} \langle \nabla^2 f(x^k) \cdot [x - x^k], x - x^k \rangle
}
\

Рассмотрим вспомогательную задачу минимизации функции $\psi_k(x)$:

\myequation{
\psi_k(x) \rightarrow \min, x \in E^n
}
\

Предположим, что решение задачи (25) $\tilde{x}^k$ существует. Очевидно, что в этом случае в точке $\tilde{x}^k$ выполняется $\nabla \psi_k (\tilde{x}^k) = 0$. Так как $\nabla \psi_k (x) = \nabla f(x^k) + \nabla^2 f(x^k) \cdot [x - x^k]$, то имеем:

\myequation{
\tilde{x}^k = x^k - [\nabla^2 f(x^k)]^{-1} \nabla f(x^k)
}
\

Таким образом в качестве направления спуска можно принять:

\myequation{
h^k = - [\nabla^2 f(x^k)]^{-1} \nabla f(x^k)
}
\

Рассмотрим точки, лежащие на отрезке $[x^k, \tilde{x}^k] : x^k(\alpha) = x^k + \alpha (\tilde{x}^k - x^k)$, где $\alpha \in [0, 1]$. Здесь заметим, что в силу (26): $\tilde{x}^k - x^k = - [\nabla^2 f(x^k)]^{-1} \nabla f(x^k)$. Конкретную точку из отрезка $[x^k, \tilde{x}^k]$ выберем, найдя из условия минимума функции $f(x^k(\alpha))$. Следующее приближение определим по формуле $x^{k+1} = x^k(\alpha_k)$. С учётом решения вспомогательной задачи (25) схема метода Ньютона примет вид:

\myequation{
\begin{cases}
x^{k+1} = x^k - \alpha_k [\nabla^2 f(x^k)]^{-1} \nabla f(x^k), \\
\alpha_k: f(x^k - \alpha_k [\nabla^2 f(x^k)]^{-1} \nabla f(x^k)) \rightarrow \min, \alpha \in [0, 1]
\end{cases}
}
\

%========================================
\mysubsection{Методы сопряженных направлений}

Рассмотри задачу минимизации квадратичной функции:

\myequation{
f(x) = \frac{1}{2} \langle Ax, x \rangle + \langle b, x \rangle \rightarrow \min, x \in E^n
}
\

где $A$ - симметричная положительно определенная $n \times n$ матрица.

Идея методов сопряженных направлений основана на стремлении найти минимум квадратичной функции (29) за конечное число шагов. Согласно методу, требуется найти направления $h^0, h^1, \ldots, h^{n-1}$ такие, что последовательная одномерная минимизация функции $f(x)$ вдоль этих направлений, начиная с любой точки $x^0 \in E^n$:

\myequation{
\begin{cases}
f(x^k + \alpha_k h^k) = \min_{\lambda_k} f(x^k + \lambda_k h^k), \\
x^{k+1} = x^k + \alpha_k h^k, (k = 0, 1, \ldots, n - 1)
\end{cases}
}
\

приводит к отысканию минимума функции (29).

\mydefinition{
Вектора $h^1$ и $h^2$ называются сопряженными (относительно матрицы $А$), если они отличны от нуля и скалярное произведение $\langle A h^1, h^2 \rangle = 0$.
}
\

\mydefinition{
Вектора $h^0, h^1, \ldots, h^{k-1}$ называются взаимно сопряженными (относительно матрицы $А$), если все они отличны от нуля и скалярное произведение $\langle A h^i, h^j \rangle = 0 \; \forall i \neq j, 0 \leq i, j \leq k$.
}
\

\mytheorem{
Пусть вектора $h^0, h^1, \ldots, h^{k-1}$ являются взаимно сопряженными, тогда они линейно не зависимы.
}
\

\myproof{
Доказательство проведем от противного. Пусть вектора $h^0, h^1, \ldots, h^{k-1}$ являются взаимно сопряженными, но при этом они являются линейно зависимыми, то есть в этом случае один из векторов можно представить в виде линейной комбинации остальных векторов, например $h^i = \sum_{j=0, j \neq i}^{k-1} \lambda_j h^j$. Тогда $\langle A h^i, h^i \rangle = \sum_{j=0, j \neq i}^{k-1} \lambda_j \langle A h^i, h^j \rangle$. И отсюда в силу взаимной сопряженной векторов $\langle A h^i, h^i \rangle = 0$, что возможно, если вектор $h^i = 0$, так как матрица является по условию симметричной положительно определенной $n \times n$ матрицей. Получили противоречие с тем, что по условию взаимной сопряженности векторов $h^i \neq 0$.
}
\

\mydefinition{
Если в методе минимизации функции $f(x)$ (29) вектора $h^0, h^1, \ldots, h^{k-1}$ взаимно сопряжены, то метод (30) называется методом сопряженных направлений
}
\

\mytheorem{
Если в методе минимизации (30) функции $f(x)$ вектора $h^0, h^1, \ldots, h^{k-1}$ взаимно сопряжены, то для функции $f(x)$, заданной формулой (29) справедливо $f(x^m) = \min_{x \in X_m} f(x)$, где $X_m = x^0 + lin \; h^0, h^1, \ldots, h^{m-1}$ линейное подпространство, натянутое на указанные векторы.
}
\

\myproof{
Предварительно заметим справедливость соотношения

\myequation{
f(x^k + \lambda_k h^k) - f(x^k) = \lambda_k \langle A x^0 + b, h^k \rangle + \frac{1}{2} \lambda_k^2 \langle A h^k, h^k \rangle
}
\

при любом $k=0, 1, \ldots, m-1$. Действительно, учитывая, что справедливо $x^k = x^0 + \sum_{i=0}^{k-1} \alpha_i h^i$ и $\langle A h^i, h^k \rangle = 0$, получим

\myequation{
f(x^k + \lambda_k h^k) = \frac{1}{2} \langle A (x^k + \lambda_k h^k), (x^k + \lambda_k h^k) \rangle + \langle b, (x^k + \lambda_k h^k) \rangle = \\
= \frac{1}{2} \langle A x^k, x^k \rangle + \langle b, x^k \rangle + \lambda_k \langle A x^k + b, h^k \rangle + \frac{1}{2} \lambda_k^2 \langle A h^k, h^k \rangle = \\
= f(x^k) + \lambda_k \langle A x^0 + \sum_{i=0}^{k-1} \alpha_i A h^i + b, h^k \rangle + \frac{1}{2} \lambda_k^2 \langle A h^k, h^k \rangle = \\
= f(x^k) + \lambda_k \langle A x^0 + b, h^k \rangle + \lambda_k \langle \sum_{i=0}^{k-1} \alpha_i A h^i , h^k \rangle + \frac{1}{2} \lambda_k^2 \langle A h^k, h^k \rangle = \\
= f(x^k) + \lambda_k \langle A x^0 + b, h^k \rangle + \frac{1}{2} \lambda_k^2 \langle A h^k, h^k \rangle
}
\

а отсюда следует справедливость (31).

Для любой точки $x \in X_m$ имеем

\myequation{
f(x) = f(x^0 + \sum_{k=0}^{m-1} \lambda_k h^k = \\
= \frac{1}{2} \langle A (x^0 + \sum_{k=0}^{m-1} \lambda_k h^k), x^0 + \sum_{k=0}^{m-1} \lambda_k h^k \rangle + \langle b, x^0 + \sum_{k=0}^{m-1} \lambda_k h^k \rangle = \\
= f(x^0) + \sum_{k=0}^{m-1} \bigg( \lambda_k \langle A x^0 + b, h^k \rangle + \frac{1}{2} \lambda_k^2 \langle A h^k, h^k \rangle \bigg)
}
\

Отсюда, с учетом (31) для любой точки $x \in X_m$ справедливо $f(x) = f(x^0) + \sum_{k=0}^{m-1} \bigg( f(x^k + \lambda_k h^k) - f(x^k) \bigg)$. Учитывая изложенное, получим

\myequation{
\min_{x \in X_m} f(x) = \min_{\lambda^0, \ldots, \lambda^{m-1}} f(x^0 + \sum_{k=0}^{m-1} \lambda_k h^k) = \\
= \min_{\lambda^0, \ldots, \lambda^{m-1}} \bigg( f(x^0) + \sum_{k=0}^{m-1} \bigg( f(x^k + \lambda_k h^k) - f(x^k) \bigg) \bigg) = \\
= f(x^0) + \sum_{k=0}^{m-1} \bigg( \min_{\lambda_k} f(x^k + \lambda_k h^k) - f(x^k) \bigg) = \\
= f(x^0) + \sum_{k=0}^{m-1} \bigg( f(x^k + \alpha_k h^k) - f(x^k) \bigg) = \\
= f(x^0) + \sum_{k=0}^{m-1} \bigg( f(x^{k+1}) - f(x^k) \bigg) = f(x^m)
}
\

Таким образом справедливо $f(x^m) = \min_{x \in X_m} f(x)$. 
}

\end{document}

